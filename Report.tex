\documentclass{article}[10pt]
\usepackage{lipsum}
\usepackage{graphicx}
\graphicspath{{images/}}
\usepackage{placeins}
\usepackage[version=4]{mhchem}
\usepackage{subfiles}
\usepackage{multicol,caption}
\usepackage{textgreek}
\usepackage{geometry}
\usepackage{url}
\newenvironment{Figure}
  {\par\medskip\noindent\minipage{\linewidth}}
  {\endminipage\par\medskip}
 \geometry{
 a4paper,
 total={170mm,257mm},
 left=20mm,
 top=20mm,
 }
\makeatletter

\makeatother
\usepackage{pgf}
\usepackage{pgfpages}

\pgfpagesdeclarelayout{boxed}
{
  \edef\pgfpageoptionborder{0pt}
}
{
  \pgfpagesphysicalpageoptions
  {%
    logical pages=1,%
  }
  \pgfpageslogicalpageoptions{1}
  {
    border code=\pgfsetlinewidth{2pt}\pgfstroke,%
    border shrink=\pgfpageoptionborder,%
    resized width=.95\pgfphysicalwidth,%
    resized height=.95\pgfphysicalheight,%
    center=\pgfpoint{.5\pgfphysicalwidth}{.5\pgfphysicalheight}%
  }%
}

\pgfpagesuselayout{boxed}
\setcounter{page}{2}
\begin{document}

  % Cover Page
  % \thispagestyle{empty}
  % \begin{center}
  % \begin{minipage}{0.75\linewidth}
  %     \centering
  %     %\includegraphics[width=0.3\linewidth]{logo.pdf}
  %     %\rule{0.4\linewidth}{0.15\linewidth}\par
  %     \vspace{3cm}
  %     {\uppercase{\Large \textbf{Multi-Wire Proportional Counter (MWPC) - A Position Sensitive Detector for Fission Fragment Detection}\par}}
  %     \vspace{3cm}
  %     {\Large \textbf{\textit{Meemik Roy}} \\ B.Sc. Hons Physics \\ Sri Venkateswara College \\ Delhi University\par}
  %     \vspace{3cm}

  %     {\Large Under the guidance of  \\ \textbf{\textit{Mr. Saneesh N}} \\ Scientist \\ Inter University Accelerator Centre (IUAC)\par}
  %     \vspace{3cm}
  %     \date{}
  % \end{minipage}
  % \end{center}
  % \clearpage


  \newpage
  \tableofcontents
  \newpage

  \section{Introduction}
    \textit{\textbf{Radiation}} is used in various fields, including medical treatment, industry, and research. 
    Radiation is a form of energy that can have both beneficial and harmful effects on living organisms.
    Therefore, it is important to accurately measure and monitor the levels of radiation to which individuals are exposed.
    Radiation exposure is a major concern in various fields, including medical treatment, industry, and research. Accurate 
    measurement and monitoring of radiation levels are crucial to prevent the harmful effects of radiation exposure.
    It is crucial to accurately measure and monitor radiation doses to ensure the safety of 
    individuals working with radiation sources and to evaluate the potential risks associated with radiation exposure. 
    \textit{\textbf{Thermoluminescence dosimetry (TLD)}} is a well-established technique that provides reliable and precise measurements 
    of ionizing radiation doses. TLDs use a variety of materials that emit light when exposed to ionizing radiation. 
    Among these materials, nanophosphors show promising properties that make them a potential candidate for use in TLDs. 
    The aim of this research is to investigate the thermoluminescent properties of synthesized nanophosphors for use in radiation dosimetry. 
    
    \subsection{\large Radiation}

      \subfile{sections/introduction/radiation}
    
    \subsection{\large Radiation Dosimetry}
      \subfile{sections/introduction/dosimetry}

    \subsection{\large Thermoluminescence Dosimetry}
      \subfile{sections/introduction/tl}

    \subsection{\large Nanophospors in Dosimetry}
      \subfile{sections/introduction/nanophosphor}

  \newpage
  \section{Synthesis and Characterization of Nanophosphors}
    The synthesis and characterization of nanophosphors for thermoluminescence dosimetry involve several steps to 
    prepare the nanocrystalline phosphor materials and evaluate their properties. Here is a general overview of 
    the process:

    \subsection{\large Synthesis Of Nanophosphors}
      \subfile{sections/synthesis/synthesis}

  \newpage
  \bibliographystyle{plain}
  \bibliography{references}
\end{document}