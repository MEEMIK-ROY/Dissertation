\documentclass[../Report.tex]{subfiles}
\graphicspath{{\subfix{../images/}}}
\begin{document}
    \vspace{1cm}
    \large
    Thermoluminescence dosimetry (TLD) is a well-established technique that provides reliable and precise measurements of ionizing radiation
    doses. TLDs use a variety of materials that emit light when exposed to ionizing radiation.Nanophosphors, also known as 
    nanocrystalline phosphors or nanoparticles, have 
    emerged as promising materials for dosimetry applications. These nanoscale phosphors exhibit unique properties 
    that make them highly suitable for radiation dosimetry.
    \\~\\
    An in-depth analysis of TLD principles as well as various techniques required for synthesis and characterization of thermoluminescent materials, 
    especially nanophosphors has been conduxted. Two nanophosphors have been synthesized using two different techniques.
    \textit{Calcium Sodium Sulphate ($CaNa_2{(SO_4)}_2$)} was synthesized using chemical co-precipitation method and 
    \textit{Lithium metasilicate ($Li_2SiO_3$)} was synthesized using solid-state reaction method their dose response 
    has been investigated. Variation in dose response due to doping has also been investigated.
    \\~\\
    The thermoluminescent properties of $CaNa_2{(SO_4)}_2$ sample was investigated by irradiating it with $^{60}Co$ 
    gamma rays at 50Gy and 100Gy.The TL glow curve of $CaNa_2{(SO_4)}_2$ shows 2 peaks at around $113^{\circ}C$ and $167^{\circ}C$ for 50 Gy
    dose and at around $109^{\circ}C$ and $166^{\circ}C$ for 100Gy dose. The sample of $CaNa_2{(SO_4)}_2:Eu(0.1mol\%)$ shows a
    very high intensity single TL glow peak at around 164◦C. This shows that the addition of dopant enhances
    the thermoluminescent properties of $CaNa_2{(SO_4)}_2$. Further analysis with different dopant concentrations is
    required to study the application of $CaNa_2{(SO_4)}_2$ as an efficient thermoluminescent dosimeter.
    \\~\\
    The thermoluminescent properties of $Li_2SiO_3$ sample was investigated by irradiating it with $^{60}Co$ 
    gamma rays at 50Gy and 100Gy.The main feature of lithium metasilicate is its high sensitivity to ionizing radiation for a wide energy range. 
    Also, $Li_2SiO_3$ ($Z_{eff} = 10.5$) being low Z materials, are near tissue equivalent and may find application in the field of 
    radiation dosimetry. Lithium metasilicate yields glow curves with two to three major peaks. Pure lithium 
    metasilicate (no dopant) has peaks at $129^{\circ}C$ and $260^{\circ}C$ at 50 Gy while it peaks at $127^{\circ}C$ 
    and $258^{\circ}C$ at 100 Gy.Lithium metasilicate doped with Europium has peaks at $164^{\circ}C$ and $267^{\circ}C$ 
    at 50 Gy while at $166^{\circ}C$ and $269^{\circ}C$ at 100 Gy.Lithium metasilicate doped with Dysprosium has peaks 
    at $132^{\circ}C$ and $258^{\circ}C$ at 50 Gy while it has 3 peaks at $134^{\circ}C$,$146^{\circ}C$ and $260^{\circ}C$ 
    at 100 Gy.Lithium metasilicate doped with Cerium has 3 peaks at $129^{\circ}C$,$162^{\circ}C$ and $263^{\circ}C$ at 
    50 Gy while and at $127^{\circ}C$,$162^{\circ}C$ and $258^{\circ}C$ at 100 Gy.Lithium metasilicate doped with Erbium 
    has peaks at $129^{\circ}C$ and $263^{\circ}C$ at 50 Gy while it has 3 peaks at $125^{\circ}C$,$151^{\circ}C$ and 
    $262^{\circ}C$ at 100 Gy.Lithium metasilicate doped with Terbium has 3 peaks at $129^{\circ}C$,$151^{\circ}C$ and 
    $260^{\circ}C$ at 50 Gy while it has 2 peaks at $132^{\circ}C$ and $262^{\circ}C$ at 100 Gy. On addition of dopants, the thermoluminescence intensities in
    much higher than that of undoped sample. Doping with Terbium and Erbium results in maximum intensities for 50 Gy and 100 Gy 
    gammma irradiation. More analysis is required in this field to provide a conclusive evidence for the use of lithium metasilicate
    as a viable and effective material for thermoluminescence dosimetry.

\end{document}