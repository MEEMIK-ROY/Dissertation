\documentclass[../Report.tex]{subfiles}
\graphicspath{{\subfix{../images/}}}
\begin{document}
    Radiation is a form of energy that can have both beneficial and harmful effects on living organisms.Accurate measurement
    and monitoring of radiation levels are crucial to prevent the harmful effects of radiation exposure.Thermoluminescence dosimetry
    (TLD) is a well-established technique that provides reliable and precise measurements of ionizing radiation
    doses. TLDs use a variety of materials that emit light when exposed to ionizing radiation.It has wide range of 
    applications in medical, archaeology, geology, meteorology and also in space dosimetry or in industries to 
    maintain the quality of products.The basic principle of thermoluminescence dosimetry involves the use of 
    thermoluminescent materials, which are crystalline or amorphous substances capable of storing energy when 
    exposed to ionizing radiation.Nanophosphors, also known as nanocrystalline phosphors or nanoparticles, have 
    emerged as promising materials for dosimetry applications. These nanoscale phosphors exhibit unique properties 
    that make them highly suitable for radiation dosimetry.

    Ideally, a TL dosimeter should exhibit the following properties: linearity in dose response, high sensitivity, 
    low fading, simple glow curve, reproducibility, tissue equivalence and stability against environmental factors 
    (the material must not undergo physical changes in high humidity, react chemically with corrosive agents or 
    show spurious thermoluminescence). Practically not all TL materials possess all of the above features. And thus, 
    researchers throughout the world constantly search for materials that could possess most of these features. In 
    the present work we have tried to prepare a TL material that possesses good number of essential features suitable 
    for radiation dosimetry. Moreover, it is also interesting to understand the physics behind electronic excitation 
    and relaxation phenomena in TL materials that are doped with dopants.

    This work explores the advancements made in thermoluminescence dosimetry, its current applications, and the 
    potential future prospects for this field. The research encompasses an in-depth analysis of TLD principles as 
    well as various techniques required for synthesis and characterization of thermoluminescent materials, 
    especially nanophosphors. In this work, two nanophosphors have been synthesized using two different techniques.
    \textit{Calcium Sodium Sulphate ($CaNa_2{(SO_4)}_2$)} was synthesized using chemical co-precipitation method and 
    \textit{Lithium metasilicate ($Li_2SiO_3$)} was synthesized using solid-state reaction method their dose response 
    was investigated. Variation in dose response due to doping was also one of the major points of focus in this research.
   
    TL Dose Response measurements indicated increase in TL intensity due to doping in both the samples. Lithium 
    metasilicate ($Li_2SiO_3$) in particular, shows two prominent peaks at distinct temperatures with increased 
    intensity on doping. This on top of being near tissue equivalent shows promise to be used as an effective material 
    to be used in dosimetry.
\end{document}