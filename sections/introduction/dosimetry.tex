\documentclass[../../Report.tex]{subfiles}
\graphicspath{{\subfix{../../images/}}}
\begin{document}
    \textbf{Radiation Dosimetry} \cite{b4} is the science and practice of measuring and assessing the dose of ionizing 
    radiation received by an object or an individual. It plays a crucial role in radiation protection, medical 
    diagnostics, and therapeutic applications. By accurately quantifying radiation doses, dosimetry enables the 
    evaluation of potential health risks, optimization of radiation procedures, and adherence to safety standards.

    \subsubsection*{\large Measurement Techniques}
        Dosimetry measurement techniques\cite{b6} are used to directly or indirectly quantify the dose of ionizing radiation 
        received by an object or an individual. These techniques vary depending on the type of radiation, the purpose 
        of measurement, and the specific application. Here are some commonly used dosimetry measurement techniques:

        \begin{itemize}
            \item \textbf{Ionization Chambers:} Ionization chambers are widely employed in radiation dosimetry. 
            They consist of a gas-filled chamber where ionization occurs when radiation interacts with the gas molecules. 
            The resulting ion pairs are collected, and the electrical current generated is proportional to the radiation 
            dose. Ionization chambers are versatile and can measure both high and low radiation doses accurately.

            \item \textbf{Thermoluminescent Dosimeters (TLDs): } TLDs are solid-state dosimeters that utilize the phenomenon of 
            thermoluminescence. When exposed to ionizing radiation, certain crystals or materials trap energy within 
            their lattice structure. Heating the TLD causes the trapped energy to be released as light, which is measured 
            and correlated to the radiation dose. TLDs offer high sensitivity and can be used for personal and environmental 
            dosimetry.

            \item \textbf{Film Dosimeters: } Film dosimeters use radiation-sensitive films, such as radiographic films or 
            photographic emulsions, to measure radiation doses. The films darken when exposed to radiation, and the degree 
            of darkening is related to the absorbed dose. Film dosimeters are commonly used in medical imaging, industrial 
            radiography, and radiation therapy verification.

            \item \textbf{Solid-State Detectors: } Solid-state detectors employ semiconductor materials, such as silicon 
            or germanium, to measure radiation doses. These detectors produce an electrical signal in response to ionizing 
            radiation. Examples of solid-state detectors include diode detectors, silicon detectors, and metal oxide 
            semiconductor field-effect transistors (MOSFETs). Solid-state detectors offer high precision, small size, and 
            real-time dose measurement capabilities.

            \item \textbf{Scintillation Detectors: } Scintillation detectors utilize scintillating materials that emit 
            light when exposed to radiation. The scintillation light is converted into electrical signals using 
            photomultiplier tubes or photodiodes. Scintillation detectors are commonly used in nuclear medicine, 
            environmental monitoring, and high-energy physics research.

            \item \textbf{Optically Stimulated Luminescence (OSL) Dosimeters: } OSL dosimeters use similar principles to 
            TLDs but employ different materials. OSL dosimeters contain optically sensitive materials that store radiation 
            energy. When exposed to light, the stored energy is released as luminescence, which is proportional to the 
            radiation dose. OSL dosimeters are widely used in personal and environmental dosimetry.

            \item \textbf{Biological Dosimeters: } Biological dosimeters measure the biological effects of radiation on 
            living cells or tissues. These dosimeters include techniques such as chromosomal aberration analysis, 
            cytogenetic biodosimetry, and biological marker assays. Biological dosimeters are particularly useful in 
            radiation accidents or emergencies to assess radiation exposure and estimate potential health risks.

            \item \textbf{Environmental Monitoring Instruments: } Environmental monitoring instruments, such as 
            Geiger-Muller counters and scintillation detectors, are used to measure ambient radiation levels in the 
            environment. These instruments provide real-time or continuous monitoring of radiation doses in occupational 
            settings, nuclear facilities, or areas with potential radioactive contamination.

        \end{itemize}
        It is worth noting that dosimetry measurement techniques are often complemented by data analysis, calibration, 
        and quality assurance procedures to ensure accurate and reliable results. The choice of dosimetry technique depends 
        on factors such as the radiation type, dose range, required sensitivity, portability, and specific dosimetry 
        application.

    \subsubsection*{\large Types of Radiation Dosimetry}
        Radiation dosimetry encompasses different types of radiation and corresponding dosimetry techniques:
        \begin{itemize}
            \item \textbf{External Beam Dosimetry: } This type of dosimetry is employed in external beam radiation 
            therapy, diagnostic imaging (e.g., X-rays, CT scans), and radiation protection. It focuses on measuring 
            radiation doses delivered from an external radiation source to a specific target or region of interest.

            \item \textbf{Internal Dosimetry: } Internal dosimetry involves the assessment of radiation doses from 
            internal radiation sources, such as radioactive materials ingested, inhaled, or injected into the body. 
            It requires the estimation of radionuclide uptake, distribution, and elimination to calculate the radiation 
            dose received by specific organs or tissues.

            \item \textbf{Environmental Dosimetry: } Environmental dosimetry aims to evaluate radiation doses in the 
            environment, particularly in occupational settings or areas with potential radioactive contamination. It 
            includes monitoring radiation levels in air, water, soil, and other environmental media to assess potential 
            exposure risks.

        \end{itemize}

    \subsubsection*{\large Applications of Radiation Dosimetry:}
        Radiation dosimetry has numerous practical applications:
        \begin{itemize}
            \item \textbf{Radiation Therapy: } In cancer treatment, radiation dosimetry is crucial for accurately 
            delivering therapeutic radiation doses to tumor sites while minimizing radiation exposure to healthy 
            tissues. Precise dosimetry measurements ensure effective treatment planning and monitoring of radiation 
            therapy procedures.

            \item \textbf{Radiation Protection: } Dosimetry is essential for ensuring the safety of workers in 
            radiation-related occupations. Occupational dosimetry involves monitoring radiation doses received by 
            individuals to ensure compliance with safety regulations and implement necessary protective measures.

            \item \textbf{Radiological Accidents and Emergency Response: } Dosimetry plays a vital role in assessing 
            radiation doses received by individuals involved in radiological accidents or emergencies. It aids in 
            triaging patients, determining appropriate medical treatments, and evaluating potential long-term health 
            effects.

            \item \textbf{Radiological Research: } Dosimetry is instrumental in research involving radiation, such as 
            studying the effects of radiation on biological systems, evaluating radiation shielding materials, and 
            optimizing radiation therapy techniques.

        \end{itemize}
    
        In summary, radiation dosimetry is an indispensable field that enables the measurement and assessment of 
        radiation doses. By employing various measurement techniques, dosimetry contributes to radiation protection, 
        medical treatments, emergency response, and scientific research. Accurate dosimetry practices are vital in 
        ensuring the safe and effective use of ionizing radiation in various applications while minimizing potential 
        health risks.

\end{document}











